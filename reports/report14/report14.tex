\documentclass{article}
\usepackage{graphicx} %package to manage images
\usepackage[utf8]{inputenc}
\usepackage[a4paper, total={6in, 8in}]{geometry}
\usepackage{xurl}
\usepackage{hyperref}
\usepackage{float}
\title{Relatório 14 \\ Verificações}
\author{Pedro A. S. O. Neto}
\date{Novembro, 2022}

\begin{document}

\maketitle

\section{Testes automatizados}

Para testar a qualidade do processamento de dados feito até este momento, foram implementadas algumas medidas automatizadas de consistência.

Para o processamento de dados fazer sentido, é necessário que:
  
\begin{itemize}
  \item O Computer.timestamp pertença a uma non-increasing monotonic function. Isto é, que apenas suba (e.g. o tempo $t+1$ não pode ser maior do que o tempo $t$)
  \item Não pode haver repetição no Computer.timestamp. Isto é, cada evento deve ser distinto no tempo, de modo que não haja, por exemplo, duas fixações ocorrendo no mesmo momento
  \item Cada Recording.name (código do participante) esteja presente no banco de dados apenas uma vez
\end{itemize}

\section{Análise automatizada}

Foram desenvolvidos os códigos necessários para a avaliação dessas assumptions. \href{https://github.com/pasoneto/EyeTracking/tree/master/code/verificacoes/assumptions}{Código disponível aqui}

\section{Resultados}

As assumptions foram verificadas e confirmadas num set de 493 participantes.

\end{document}
