\documentclass{article}
\usepackage{graphicx} %package to manage images
\usepackage[utf8]{inputenc}
\usepackage[a4paper, total={6in, 8in}]{geometry}
\usepackage{xurl}
\usepackage{hyperref}
\usepackage{float}
\title{Relatório 17 \\ Proporções e critério de inclusão (Todos os participantes)}
\author{Pedro A. S. O. Neto}
\date{Dezembro, 2022}

\begin{document}

\maketitle

\section{Critérios de inclusão e resultados}

Reporto todas as filtragens realizadas ao longo do pipeline de processamento dos dados.

\section{Tempo relativo de fixação}


Como primeira análise, computamos o tempo relativo que cada criança passou olhando para cada objeto na tela. Haviam 4 possibilidades:

\begin{itemize}
  \item Rosto
  \item Fundo
  \item Objeto target (direita ou esquerda) 
  \item Objeto distrator (direita ou esquerda)
\end{itemize}

Desta forma, cada proporção indica quanto tempo a criança passou olhando para o objeto $x$, em relação ao tempo total que a criança passou olhando para a tela. Resultados na Figura 1 e na Tabela 2.

\begin{table}[ht]
\centering
\begin{tabular}{rrrr}
  \hline
  Objeto & Condição & Diagnóstsico & Proporção relativa \\
  \hline
   & IJA & TEA &  \\
   & RJA & TD &  \\
   & IJA & TEA &  \\
   & RJA & TD &  \\
  \hline
\end{tabular}
\end{table}

\end{document}
