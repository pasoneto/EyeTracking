\documentclass{article}
\usepackage{graphicx} %package to manage images
\usepackage[utf8]{inputenc}
\usepackage[a4paper, total={6in, 8in}]{geometry}
\usepackage{xurl}
\usepackage{hyperref}
\usepackage{float}
\title{Relatório 4 \\ Permutation test}
\author{Pedro A. S. O. Neto}
\date{Maio, 2023}

\begin{document}

\maketitle

\section{Replicação - Pierce}

\subsection{Sample}

\begin{itemize}
  \item N = 334; 
  \item 6 diagnostics = 115 ASD, 20 ASD-Feat, 57 DD, 53 Other, 64 TD, and 25 Typ SIB
  \item N=444; 152 ASD, 20 ASD-Feat, 79 DD, 53 Other, 115 TD, and 25 Typ SIB
\end{itemize}

\subsection{Analysis}

\begin{itemize}
  \item 35-pixel radius filter. 
  \item Percent time spent fixating within each AOI (i.e., DGI or DSI) was tabulated for each subject.
  \item Compare percent fixation time within DGI between groups a one-way analysis of covariance (ANCOVA) was performed with 6 levels (diagnostic groups) using the age of the child at testing as a covariate.
  \item Significant effects were followed by planned tests with Bonferroni correction for multiple comparisons
  \item Relationship between percent fixation on DGI and clinical measures were determined using linear regression controlling for the effects of age
  \item Threshold of percentage of fixation time within DGI that would best discriminate ASD from other toddlers, a receiver operating curve (ROC) was generated
  \item PPV and NPV (positive and negative predictive vallue) tested in two different ways: 1) Were calculated within the study sample, as would be applicable in a 2nd tier screening approach; 2) PPV and NPV were calculated taking into account the ASD prevalence rate of 1.47 percent in the general population (27), as would be applicable in a 1st tier screening approach.
  \item The number of saccades per second was determined for each subject by dividing the overall total number of saccades by the total looking time. Differences in saccade data between diagnostic groups were examined utilizing a one-way ANCOVA covarying for age and planned contrasts.
\end{itemize}



\end{document}




